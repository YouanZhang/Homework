\documentclass[12pt]{ctexart}
\usepackage{listings}
\usepackage{amsmath} 
\usepackage{amssymb} 
\usepackage{xcolor}
\usepackage{xeCJK}
\usepackage{fontspec}
\usepackage{titlesec}
\usepackage{titletoc}
\usepackage{setspace}
\usepackage{graphicx}
\usepackage{geometry}
\usepackage[T1]{fontenc}  
\usepackage{textcomp}  
\usepackage{lmodern}
\usepackage[colorlinks,
            linkcolor=black,
            anchorcolor=black,
            citecolor=black]{hyperref}
\geometry{a4paper,scale=0.8}
\renewcommand\contentsname{Contents}
%\setmonofont[Mapping={}]{Monaco}    %英文引号之类的正常显示,相当于设置英文字体
%\setsansfont{Consolas} %设置英文字体 Monaco, Consolas,  Fantasque Sans Mono
%\setmainfont{Monaco} %设置英文字体
\setmonofont{Consolas}
% 定义可能使用到的颜色
%\setmainfont[BoldFont=SimHei]{SimSun}
\definecolor{CPPLight}  {HTML} {686868}
\definecolor{CPPSteel}  {HTML} {888888}
\definecolor{CPPDark}   {HTML} {262626}
\definecolor{CPPBlue}   {HTML} {4172A3}
\definecolor{CPPGreen}  {HTML} {487818}
\definecolor{CPPBrown}  {HTML} {A07040}
\definecolor{CPPRed}    {HTML} {AD4D3A}
\definecolor{CPPViolet} {HTML} {7040A0}
\definecolor{CPPGray}  {HTML} {B8B8B8}
\lstset{
    columns=fixed,       
    % numbers=left,                                        % 在左侧显示行号
    frame=none,                                          % 不显示背景边框
    backgroundcolor=\color[RGB]{245,245,244},            % 设定背景颜色
    keywordstyle=\color[RGB]{40,40,255},                 % 设定关键字颜色
    numberstyle=\small\color{darkgray},                  % 设定行号格式
    commentstyle=\it\color[RGB]{0,96,96},                % 设置代码注释的格式
    stringstyle=\rmfamily\slshape\color[RGB]{128,0,0},   % 设置字符串格式
    showstringspaces=false,                              % 不显示字符串中的空格
    language=c++,                                        % 设置语言
    morekeywords={alignas,continute,friend,register,true,alignof,decltype,goto,
    reinterpret_cast,try,asm,defult,if,return,typedef,auto,delete,inline,short,
    typeid,bool,do,int,signed,typename,break,double,long,sizeof,union,case,
    dynamic_cast,mutable,static,unsigned,catch,else,namespace,static_assert,using,
    char,enum,new,static_cast,virtual,char16_t,char32_t,explict,noexcept,struct,
    void,export,nullptr,switch,volatile,class,extern,operator,template,wchar_t,
    const,false,private,this,while,constexpr,float,protected,thread_local,
    const_cast,for,public,throw,std},
    emph={map,set,multimap,multiset,unordered_map,unordered_set,
    unordered_multiset,unordered_multimap,vector,string,list,deque,
    array,stack,forwared_list,iostream,memory,shared_ptr,unique_ptr,
    random,bitset,ostream,istream,cout,cin,endl,move,default_random_engine,
    uniform_int_distribution,iterator,algorithm,functional,bing,numeric,},
    emphstyle=\color{CPPViolet},
    basicstyle=\linespread{1}\small\fontspec{Consolas}\ttfamily,
    breaklines=true,
    %xleftmargin=1em,xrightmargin=1em, aboveskip=1em,
    % in the listings package configuration, try:  
    literate={"}{\textquotedbl}1,  
    tabsize=4, keepspaces=true
}
\CTEXoptions[today=old]
\title{Exercises 4}
\author{张逸松}
\date{\today}
\begin{document}
    \maketitle
    \subsection*{4.4}
    \begin{lstlisting}
virtual bool interChange() {
    int curPos = currPos();
    E& temp = getValue();
    remove();
    next();
    if (curPos == currPos()) return false;
    prev();
    insert(temp);
}
    \end{lstlisting}

    \subsection*{4.6}
    \begin{lstlisting}
void reverse() {
    if (head == tail) return;
    head = head->next; append(0);
    moveToStart();
    for (Link<E>* temp = curr->next; curr != NULL; curr = temp, temp = temp->next)
        curr->next = curr;
    Link<E> temp = head;
    head = tail;
    tail = temp;
}
    \end{lstlisting}

    \subsection*{4.13}
    \begin{lstlisting}
template <typename E> class TwoStack: public Stack<E> {
private:
    int maxSize;
    int top1, top2;
    E* listArray;
public:
    TwoStack(int size = defaultSize) {
        maxSize = size; top1 = 0; top2 = size - 1; listArray = new E[size];
    }
    ~TwoStack() { delete [] listArray; }
    void clear(){ top1 = 0; top2 = maxSize - 1; }
    void push1(const E& it){
        Assert(top1 != top2, "Stack is full");
        listArray[top1++] = it;
    }
    void push2(const E& it){
        Assert(top1 != top2, "Stack is full");
        listArray[top2--] = it;
    }
    E pop1() {
        Assert(top1 != 0, "Stack1 is empty");
        return listArray[--top1];
    }
    E pop2() {
        Assert(top2 != maxSize - 1, "Stack2 is empty");
        return listArray[++top2];
    }
    const E& top1Value() {
        Assert(top1 != 0, "Stack1 is empty");
        return listArray[top1 - 1];
    }
    const E& top2Value() {
        Assert(top2 != maxSize - 1, "Stack2 is empty");
        return listArray[top2 + 1];
    }
    int length() const { return top1 + maxSize - 1 - top2; }
};
    \end{lstlisting}

    \subsection*{4.14}
    \begin{lstlisting}
template <typename E> class AQueue: public Queue<E> {
private:
    int maxSize;
    int front;
    int rear;
    bool empty;
    E* listArray;
public:
    AQueue(int size = defaultSize) {
        maxSize = size + 1;
        rear = 0; front = 1;
        empty = 1;
        listArray = new E[maxSize];
    }
    ~AQueue() { delete [] listArray; }
    void clear() { rear = 0; front = 1; empty = 1; }
    void enqueue(const E& it) {
        Assert(((rear + 2) % maxSize) != front, "Queue is full");
        rear = (rear + 1) % maxSize;
        listArray[rear] = it;
        empty = 0;
    }
    E dequeue() {
        Assert(empty != 1, "Queue is empty");
        E it = listArray[front];
        front = (front + 1) % maxSize;
        if ((rear + 1) % maxSize == front) empty = 1;
        return it;
    }
    const E& frontValue() const {
        Assert(empty != 1, "Queue is empty");
        return listArray[front];
    }
};
    \end{lstlisting}

    \subsection*{4.15}
    \begin{lstlisting}
bool isPalindrome() {
    Stack <char> S;
    Queue <char> Q;
    char c;
    for (; cin >> c; ) {
        S.push(c);
        Q.enqueue(c);
    }
    for (; Q.length(); ) {
        if (S.top() != Q.front()) return false;
        char temp = S.pop();
        temp = Q.dequeue();
    }
    return true;
}
    \end{lstlisting}

    \subsection*{4.19}
    \subsubsection*{a)}
    \begin{lstlisting}
bool isBalanced(string str) {
    stack <int> s;
    for (int i = 0; i < str.size(); ++i) {
        if (str[i] == '(') s.push(i);
        else if (str[i] == ')') {
            if (s.empty()) return false;
            else s.pop();
        }
    }
    return s.empty();
}
    \end{lstlisting}

    \subsubsection*{b)}
    \begin{lstlisting}
int FOP(string str) {
    stack <int> s;
    for (int i = 0; i < str.size(); ++i) {
        if (str[i] == '(') s.push(i);
        else if (str[i] == ')') {
            if (s.empty()) return i;
            else s.pop();
        }
    }
    if (s.empty()) return -1;
    for (int i = s.top(); ; i = s.top(), s.pop())
        if (s.empty()) return i;
}
    \end{lstlisting}
\end{document}