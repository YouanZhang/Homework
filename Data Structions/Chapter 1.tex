\documentclass[12pt]{ctexart}
\usepackage{listings}
\usepackage{amsmath} 
\usepackage{amssymb} 
\usepackage{xcolor}
\usepackage{xeCJK}
\usepackage{fontspec}
\usepackage{titlesec}
\usepackage{titletoc}
\usepackage{setspace}
\usepackage{graphicx}
\usepackage{geometry}
\usepackage[T1]{fontenc}  
\usepackage{textcomp}  
\usepackage{lmodern}
\usepackage[colorlinks,
            linkcolor=black,
            anchorcolor=black,
            citecolor=black]{hyperref}
\geometry{a4paper,scale=0.8}
\renewcommand\contentsname{Contents}
%\setmonofont[Mapping={}]{Monaco}    %英文引号之类的正常显示,相当于设置英文字体
%\setsansfont{Consolas} %设置英文字体 Monaco, Consolas,  Fantasque Sans Mono
%\setmainfont{Monaco} %设置英文字体
\setmonofont{Consolas}
% 定义可能使用到的颜色
%\setmainfont[BoldFont=SimHei]{SimSun}
\definecolor{CPPLight}  {HTML} {686868}
\definecolor{CPPSteel}  {HTML} {888888}
\definecolor{CPPDark}   {HTML} {262626}
\definecolor{CPPBlue}   {HTML} {4172A3}
\definecolor{CPPGreen}  {HTML} {487818}
\definecolor{CPPBrown}  {HTML} {A07040}
\definecolor{CPPRed}    {HTML} {AD4D3A}
\definecolor{CPPViolet} {HTML} {7040A0}
\definecolor{CPPGray}  {HTML} {B8B8B8}
\lstset{
    columns=fixed,       
    % numbers=left,                                        % 在左侧显示行号
    frame=none,                                          % 不显示背景边框
    backgroundcolor=\color[RGB]{245,245,244},            % 设定背景颜色
    keywordstyle=\color[RGB]{40,40,255},                 % 设定关键字颜色
    numberstyle=\small\color{darkgray},                  % 设定行号格式
    commentstyle=\it\color[RGB]{0,96,96},                % 设置代码注释的格式
    stringstyle=\rmfamily\slshape\color[RGB]{128,0,0},   % 设置字符串格式
    showstringspaces=false,                              % 不显示字符串中的空格
    language=c++,                                        % 设置语言
    morekeywords={alignas,continute,friend,register,true,alignof,decltype,goto,
    reinterpret_cast,try,asm,defult,if,return,typedef,auto,delete,inline,short,
    typeid,bool,do,int,signed,typename,break,double,long,sizeof,union,case,
    dynamic_cast,mutable,static,unsigned,catch,else,namespace,static_assert,using,
    char,enum,new,static_cast,virtual,char16_t,char32_t,explict,noexcept,struct,
    void,export,nullptr,switch,volatile,class,extern,operator,template,wchar_t,
    const,false,private,this,while,constexpr,float,protected,thread_local,
    const_cast,for,public,throw,std},
    emph={map,set,multimap,multiset,unordered_map,unordered_set,
    unordered_multiset,unordered_multimap,vector,string,list,deque,
    array,stack,forwared_list,iostream,memory,shared_ptr,unique_ptr,
    random,bitset,ostream,istream,cout,cin,endl,move,default_random_engine,
    uniform_int_distribution,iterator,algorithm,functional,bing,numeric,},
    emphstyle=\color{CPPViolet},
    basicstyle=\linespread{1}\small\fontspec{Consolas}\ttfamily,
    breaklines=true,
    %xleftmargin=1em,xrightmargin=1em, aboveskip=1em,
    % in the listings package configuration, try:  
    literate={"}{\textquotedbl}1,  
    tabsize=4, keepspaces=true
}
\CTEXoptions[today=old]
\title{Exercises 1}
\author{张逸松}
\date{\today}


\begin{document}
    \maketitle
    \subsection*{1.3}
        \subsubsection*{My functions}
        \begin{lstlisting}
// Copy s1 to s2 
ADT_String strcpy(ADT_String s1, ADT_String s2);

// Concatenate two strings
ADT_String strcat(ADT_String s1, ADT_String s2);

// Return the length of s
int strlen(ADT_String s);

// Intercept the substring from s, starting at 'pos', and of length of 'size'
ADT_String substr(ADT_String s, int pos, int size);

// Return 0 (if s1 is equal to s2), 1 (if s1 is greater then s2), -1 (if s1 is less than s2)
int strcmp(ADT_String s1, ADT_String s2);

// Find the location where c first appeared in s (0-based)
int strchr(ADT_String s, char c);

// Reverse s
ADT_String strrev(ADT_String s);
        \end{lstlisting}

        \subsubsection*{First physical representation}
        \begin{lstlisting}
class ADT_String
{
    vector <char> s;
};
        \end{lstlisting}

        \subsubsection*{Second physical representation}
        \begin{lstlisting}
class ADT_String
{
    char c;
    ADT_String *nxt;
};
        \end{lstlisting}

    \subsection*{1.4}
        \subsubsection*{An ADT for a list of integers might specify the following operations:}
            \begin{itemize}
                \item Insert a new integer at a particular position in the list.
                \item Return \textbf{true} if the list is empty.
                \item Reinitialize the list.
                \item Return the number of integers currently in the list.
                \item Delete the integer at a particular position in the list.
            \end{itemize}

        \subsubsection*{Code}
        \begin{lstlisting}
class ADT_List_Integer
{
private:
    int value;
    ADT_List_integer* nxt;
protected:
    virturl void push_back(int x);
    virturl bool empty();
    virturl void clear();
    virturl int size();
    virturl void delete();
};
        \end{lstlisting}
    
    \subsection*{1.6}
        \begin{lstlisting}
ADT add(ADT a, ADT b);
ADT multiply(ADT a, ADT b);
ADT transpose(ADT a);
void setValue(ADT a, int x, int y, int value);
int getValue(ADT a, int x, int y);
        \end{lstlisting}

        \subsubsection*{Implementation}
        One implementation is a two-dimensional hash table, another one is the adjacency table.

    \subsection*{1.7}
        \begin{lstlisting}
funcion BublleSort(A)
    n <- the length of A
    k <- n - 1
        for step = 1 to k do
            for i = 2 to n do
                if A[i - 1] > A[i] then
                    swap A[i - 1] and A[i]
        \end{lstlisting}

    \subsection*{1.8}
        \begin{lstlisting}
map <ADT1, ADT2> mp;
ADT1 key; ADT2 value;
auto it = mp.find(key);
it->second = value;
        \end{lstlisting}
\end{document}