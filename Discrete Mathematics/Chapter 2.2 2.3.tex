\documentclass[10pt]{ctexart}
\usepackage{listings}
\usepackage{amsmath} 
\usepackage{amssymb} 
\usepackage{xcolor}
\usepackage{xeCJK}
\usepackage{fontspec}
\usepackage{titlesec}
\usepackage{titletoc}
\usepackage{setspace}
\usepackage{graphicx}
\usepackage{geometry}
\usepackage[T1]{fontenc}  
\usepackage{textcomp}
\usepackage{lmodern}
%\usepackage{caption}
\usepackage[justification=centering]{subcaption}
\usepackage[justification=centering]{caption}
\usepackage[colorlinks,
            linkcolor=black,
            anchorcolor=black,
            citecolor=black]{hyperref}
\geometry{a4paper,scale=0.8}
\renewcommand\contentsname{Contents}
%\setmonofont[Mapping={}]{Monaco}    %英文引号之类的正常显示,相当于设置英文字体
%\setsansfont{Consolas} %设置英文字体 Monaco, Consolas,  Fantasque Sans Mono
%\setmainfont{Monaco} %设置英文字体
\setmonofont{Consolas}
% 定义可能使用到的颜色
%\setmainfont[BoldFont=SimHei]{SimSun}
\definecolor{CPPLight}  {HTML} {686868}
\definecolor{CPPSteel}  {HTML} {888888}
\definecolor{CPPDark}   {HTML} {262626}
\definecolor{CPPBlue}   {HTML} {4172A3}
\definecolor{CPPGreen}  {HTML} {487818}
\definecolor{CPPBrown}  {HTML} {A07040}
\definecolor{CPPRed}    {HTML} {AD4D3A}
\definecolor{CPPViolet} {HTML} {7040A0}
\definecolor{CPPGray}  {HTML} {B8B8B8}
\lstset{
    columns=fixed,       
    % numbers=left,                                        % 在左侧显示行号
    frame=none,                                          % 不显示背景边框
    backgroundcolor=\color[RGB]{245,245,244},            % 设定背景颜色
    keywordstyle=\color[RGB]{40,40,255},                 % 设定关键字颜色
    numberstyle=\small\color{darkgray},                  % 设定行号格式
    commentstyle=\it\color[RGB]{0,96,96},                % 设置代码注释的格式
    stringstyle=\rmfamily\slshape\color[RGB]{128,0,0},   % 设置字符串格式
    showstringspaces=false,                              % 不显示字符串中的空格
    language=c++,                                        % 设置语言
    morekeywords={alignas,continute,friend,register,true,alignof,decltype,goto,
    reinterpret_cast,try,asm,defult,if,return,typedef,auto,delete,inline,short,
    typeid,bool,do,int,signed,typename,break,double,long,sizeof,union,case,
    dynamic_cast,mutable,static,unsigned,catch,else,namespace,static_assert,using,
    char,enum,new,static_cast,virtual,char16_t,char32_t,explict,noexcept,struct,
    void,export,nullptr,switch,volatile,class,extern,operator,template,wchar_t,
    const,false,private,this,while,constexpr,float,protected,thread_local,
    const_cast,for,public,throw,std},
    emph={map,set,multimap,multiset,unordered_map,unordered_set,
    unordered_multiset,unordered_multimap,vector,string,list,deque,
    array,stack,forwared_list,iostream,memory,shared_ptr,unique_ptr,
    random,bitset,ostream,istream,cout,cin,endl,move,default_random_engine,
    uniform_int_distribution,iterator,algorithm,functional,bing,numeric,},
    emphstyle=\color{CPPViolet},
    basicstyle=\linespread{1}\small\fontspec{Consolas}\ttfamily,
    breaklines=true,
    %xleftmargin=1em,xrightmargin=1em, aboveskip=1em,
    % in the listings package configuration, try:  
    literate={"}{\textquotedbl}1,  
    tabsize=4, keepspaces=true
}
\CTEXoptions[today=old]
\title{Exercises 2}
\author{软件工程一班 \ 张逸松 57号}
\date{\today}

\begin{document}
    \maketitle
    \subsection*{2.9}
    \subsubsection*{a)}
    $x \in \overline{A \cap B \cap C}
    \equiv x \notin A \cup B \cup C 
    \equiv x \notin A \vee x \notin B \vee x \notin C
    \equiv x \in \overline{A} \vee x \in \overline{B} \vee x \in \overline{C}
    \equiv x \in \overline{A} \cup \overline{B} \cup \overline{C}$
    \subsubsection*{b)}
    \begin{table}[h]
        \begin{tabular}{|c|c|c|c|c|}
            \hline
            $A\ B\ C$ & $A \cap B \cap C$ & $\overline{A \cap B \cap C}$ & $\overline{A}\ \overline{B}\ \overline{C}$ & $\overline{A} \cup \overline{B} \cup \overline{C}$\\ 
            \hline
            $0\ 0\ 0$ & $0$ & $1$ & $1\ 1\ 1$ & $1$ \\ 
            \hline
            $0\ 0\ 1$ & $0$ & $1$ & $1\ 1\ 1$ & $1$ \\ 
            \hline
            $0\ 1\ 0$ & $0$ & $1$ & $1\ 1\ 1$ & $1$ \\ 
            \hline
            $0\ 1\ 1$ & $0$ & $1$ & $1\ 1\ 1$ & $1$ \\ 
            \hline
            $1\ 0\ 0$ & $0$ & $1$ & $1\ 1\ 1$ & $1$ \\ 
            \hline
            $1\ 0\ 1$ & $0$ & $1$ & $1\ 1\ 1$ & $1$ \\ 
            \hline
            $1\ 1\ 0$ & $0$ & $1$ & $1\ 1\ 1$ & $1$ \\ 
            \hline
            $1\ 1\ 1$ & $1$ & $0$ & $1\ 1\ 1$ & $0$ \\ 
            \hline
        \end{tabular}
    \end{table}
    \subsection*{2.12}
    $x \in A \cup (B \cap C)
    \equiv x \in A \vee x \in (B \cap C)
    \equiv x \in A \vee (x \in B \wedge x \in C)
    \equiv (x \in A \vee x \in B) \wedge (x \in A \vee x \in C)
    \equiv x \in (A \cup B) \cap (A \cup C)$
    \subsection*{2.13}
    \textbf{a)} $A \cap B \cap C = \left\{4,6\right\}$ \par
    \textbf{b)} $A \cap B \cap C = \left\{0,1,2,3,4,5,6,7,8,9,10\right\}$ \par
    \textbf{c)} $A \cap B \cap C = \left\{4,5,6,8,10\right\}$ \par
    \textbf{d)} $A \cap B \cap C = \left\{0,2,4,5,6,7,8,9,10\right\}$
    \subsection*{3.1}
    \textbf{a)} $f(0)$ is not defined. \par
    \textbf{b)} $f(x)$ is not defined when $x < 0$. \par
    \textbf{c)} $f(x)$ has two value assigned to x.
    \textbf{a)} $\left\lceil \frac{3}{4} \right\rceil = 1 $ \quad
    \textbf{b)} $\left\lfloor \frac{7}{8} \right\rfloor = 0 $ \quad
    \textbf{c)} $\left\lceil -\frac{3}{4} \right\rceil = 0 $ \quad
    \textbf{d)} $\left\lfloor -\frac{7}{8} \right\rfloor = -1 $ \quad
    \par
    \textbf{e)} $\left\lceil 3 \right\rceil = 3 $ \quad
    \textbf{f)} $\left\lfloor -1 \right\rfloor = -1 $ \quad
    \textbf{g)} $\left\lfloor \frac{1}{2} + \left\lceil \frac{3}{2}\right\rceil \right\rfloor = 2 $ \quad
    \textbf{h)} $\left\lfloor \frac{1}{2} \cdot \left\lfloor \frac{5}{2}\right\rfloor \right\rfloor = 1 $ \quad
    \subsection*{3.7}
    \textbf{a)} $f(n) = n - 1$ is a onto. \par
    \textbf{b)} $f(n) = n^2 + 1$ is not a onto. \par
    \textbf{c)} $f(n) = n^3$ is a onto. \par
    \textbf{d)} $f(n) = \left\lfloor n/2 \right\rfloor$ is a onto.
    \subsection*{3.12}
    \textbf{a)} $f(x) = 2x + 1$ is a bijection. \par
    \textbf{b)} $f(x) = x^2 + 1$ is not a bijection. \par
    \textbf{c)} $f(x) = x^3$ is a bijection. \par
    \textbf{d)} $f(x) = (x^2 + 1)/(x^2 + 2)$ is a bijection.
    \subsection*{3.16}
    \textbf{a)} $f(S) = \{0,1,3\}$.\par
    \textbf{b)} $f(S) = \{0,1,3,5,8\}$.\par
    \textbf{c)} $f(S) = \{0,8,16,40\}$.\par
    \textbf{d)} $f(S) = \{1,12,33,65\}$.

\end{document}